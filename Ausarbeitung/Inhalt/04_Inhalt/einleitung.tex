\chapter{Einleitung}

Die Evaluierung von Machine-Learning-Modellen erfolgt anhand geeigneter
Metriken. Dabei hängt die Wahl der Metriken vom
spezifischen Aufgabentyp ab (Clustering, Klassifikation oder Regression). Gute Metriken
legen fest, wie Qualität im jeweiligen Kontext gemessen wird. So strebt man
beim Clustering dichte, gut getrennte Cluster an, bei der Klassifikation eine
hohe Übereinstimmung mit den Klassenlabels, und bei der Regression geringe
Abweichungen der Vorhersagen vom wahren Wert. \cite{Miller2024} legt dar,
dass das Verständnis und die richtige Auswahl von Metriken entscheidend sind,
um die Modellleistung objektiv zu beurteilen und Fehlinterpretationen zu
vermeiden. 

Die vorliegende Arbeit soll einen Überblick über die wichtigsten Evaluationsmetriken in diesen Zusammenhang liefern.
Sie soll dabei als Grundlage dienen, welche Evaluationsmetriken für die verschiedenen ML-Algorithmen verwendet werden können.

Im Folgenden werden die wichtigsten Metriken für Clustering, Klassifikation und Regression vorgestellt
und hinsichtlich ihrer Stärken und Schwächen diskutiert.

